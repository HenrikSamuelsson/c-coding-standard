\documentclass{article}

\usepackage{booktabs}       % For more professional looking tables.
\usepackage{listings}       % For displaying source code.
\usepackage{inconsolata}    % Enables usage of font that mimics Consolas font.
\usepackage{tabularx}       % Enables stretching table column width to fit page.

\lstset{
  basicstyle=\ttfamily,     % Change to mono spaced font for source code.
  tabsize=4,                % Reduce the tab size.
  language=C,               % Set C as programming language.
  frame=single              % Put a frame around the source code listings.
}

\newcounter{Rule}[section]
\newcommand\myRule{\stepcounter{Rule}Rule \theRule}

\begin{document}

\section{Introduction}

\subsection{Terms and Definitions}

A quick reference with explanations of terms used in this document is provided in Table \ref{tab:terms-and-defs}. Note that the definitions are often C programming language specific and might not be true in other contexts.

\begin{table}[h]
\begin{tabularx}{\textwidth}{lX}
\toprule
Term & Definition \\
\midrule
allocated storage duration & Object with dynamic storage duration, the storage is created and destroyed upon request. \\
automatic	storage duration &	Object with storage duration inside a block only, from the point of declaration to the end of the block. \\
block & A range of statements enclosed between a pair of braces. \\
declaration & Statement that introduces an identifier and its type. \\
identifier & A name given to an entity such as variables, functions, structures, et cetera. \\
object & A location in memory whose content represent a value. \\
statement & Instruction to a computer, in C typically a single line expression followed by a semicolon. \\
static storage duration & Storage duration from start to end of program execution. \\
storage duration & Determines an objects lifetime. There are three storage durations; allocated, auto, and static. \\
storage class specifier & Specifies storage duration and linkage of objects. There are five specifiers; auto, extern, register, static, and \_Thread\_local. \\
type & Variable classification such as; int, char, double, et cetera, that determines storage size and how the bit pattern stored shall be interpreted.\\
type qualifier & Adds attributes to types at the point of declaration. There are four type qualifiers; const, restrict, volatile, and \_Atomic. \\
\bottomrule
\end{tabularx}
\caption{Term and definition quick reference}
\label{tab:terms-and-defs}
\end{table}

\section{Rules}

\subsection{Initialize Automatic Variables}

Variables with automatic storage duration shall be initialized before use.

\subsubsection{Reasoning}

The value of an automatic variable that is not initialized is undefined. Explicit initialization shall hence be done before using the variable, either when declaring the variable or just before the variable is used for the first time.
\begin{minipage}[t]{0.47\linewidth}
\begin{lstlisting}[frame=single, title={Non-Compliant}]
int foo(void)
{
	uint32_t i;
	i++;
	return i;
}
\end{lstlisting}
\end{minipage}
\hfill
\begin{minipage}[t]{0.47\linewidth}
\begin{lstlisting}[title={Compliant}]
int foo(void)
{
	uint32_t i = 0;
	i++;
	return i;
}
\end{lstlisting}
\end{minipage}

\subsection{Initialize Constant Variables When Declared}

Declaration of variables with const type qualifier shall include initialization.

\subsection{Strict Enumeration Initialization}

Strategy used for initialization of the members in an enum type shall be one of the following: not specifying any values, specifying all values, specifying only the first value.

\subsubsection{Reasoning}

Minimizes the risk that a pair of members is assigned the same value by mistake.
\begin{minipage}[t]{0.47\linewidth}
\lstinputlisting[firstline=25, lastline=30, title={Non-Compliant}]{../code/main.c}
\end{minipage}
\hfill
\begin{minipage}[t]{0.47\linewidth}
\lstinputlisting[firstline=4, lastline=23, title={Compliant}]{../code/main.c}
\end{minipage}

\subsection{No Usage of Restrict}

Do not use the restrict type qualifier.

\subsubsection{Reasoning}

The keyword restrict is type qualifier that can be added in a object pointer declaration. It provides a hint to the compiler that only this pointer will be used access the object. This will in some situations make it possible for the compiler to generate a more optimized result. The behavior of the code will be undefined if this guarantee is not meet. Using restrict burdens the design of the code to guarantee that the memory areas do not overlap and adds a risk, the restrict type qualifier shall hence not be used.

\subsubsection{Examples}

\begin{minipage}[t]{0.47\linewidth}
\lstinputlisting[firstline=40, lastline=46, title={Non-Compliant}]{../code/main.c}
\end{minipage}
\hfill
\begin{minipage}[t]{0.47\linewidth}
\lstinputlisting[firstline=32, lastline=38, title={Compliant}]{../code/main.c}
\end{minipage}

\subsection{Use Unsigned Type For Bit Manipulation}

Operands of bitwise operation shall be of unsigned integer type.

\subsubsection{Reasoning}

The result of bitwise operations on signed integers are implementation-defined.

\subsection*{\myRule{}}

The declaration of a function that does not take any parameters shall use the void type parameter.

\subsubsection*{Reasoning}

A C function declaration with an empty parameter list is not the same as that the function has no parameters, it is an obsolete way to declare a function without needing to explicitly specify the number, and types of parameters. Using void states explicitly that a function does not takes any parameters, making it possible for the compiler to check for conflicts in the function usage.

\subsubsection*{Examples}

\begin{minipage}[t]{0.47\linewidth}
\lstinputlisting[firstline=3, lastline=3, title={Non-Compliant}]{../code/rule_001_non_compliant.c}
\end{minipage}
\hfill
\begin{minipage}[t]{0.47\linewidth}
\lstinputlisting[firstline=3, lastline=3, title={Compliant}]{../code/rule_001_compliant.c}
\end{minipage}

\subsection{Related Constants Definition}

Us an enum to define related constants.

\subsection{Examples}

TODO Add code examples.
TODO Check if GCC spots the problem in the non-compliant code example.

\subsubsection{Reasoning}

Defining related constants as enum type, as opposed to a group of preprocessor defines, makes it possible to automate checks for errors where the constant is used in the wrong context.

Debugging becomes simplified due to that there will be a symbol for each enum constant in the debugger symbol table.

\subsection*{\myRule{}}

Ensure that pointer cast preserves the type qualifiers of the type addressed by the pointer.

\subsubsection*{Examples}

TODO fix the other Example

\noindent
\begin{minipage}[t]{\codelstwidth\linewidth}
    \lstinputlisting[
        firstline=3,
        lastline=12,
        title={Non-Compliant}
        ]{../code/rule_003_non_compliant.c}
\end{minipage}
\hfill
\begin{minipage}[t]{\codelstwidth\linewidth}
    \lstinputlisting[
        firstline=3,
        lastline=3,
        title={Compliant}
        ]{../code/rule_001_compliant.c}
\end{minipage}

\subsubsection*{Reasoning}

Casting away const and volatile memory area qualifications is not illegal but adds some risks. The compiler will not be able to check and detect erroneous handling of the memory area. The compiler might also perform unintended optimization.

\section*{Appendix A Doxygen}

\subsection*{Introduction}

Doxygen is a tool that generates documentation from annotated source files. A software module can be documented directly when developing the code for the module. This workflow makes it more likely that the documentation is kept consistent with the source code.

The documentation can be generated into various formats depending on the needs of the project process. Examples of supported formats are PDF, HTML, RTF, and \LaTeX{}.

Doxygen is commonly used to document modules and functions intended usage in a textual format. But is also possible to generate various visual representations of the elements in the form of graphs and diagrams. These can be used learn the structure of a project or to verify that the implementation is consistent with the design.

The annotations added to the source code are mostly straightforward and human readable. It is possible for developers to digest the documentation directly in the raw source code without need to run the generation process.

\subsection*{Source Code Annotation Rules}

The Doxygen specific annotations are created by incorporating special kind of comment blocks into the source files. These comment blocks can be written in several different ways and still be accepted by Doxygen. This section specifies a set of rules for how the write the comment blocks. The purpose of the rules is; to have consistency, not miss out on required documentation, trouble free coexistence with other tools.

\subsection*{\doxygenRule{}}

Mark Doxygen comment blocks as a C-style comment block, with the difference that there shall be two initial asterisks.

\noindent
\begin{minipage}{0.47\textwidth}
    \lstinputlisting[
        firstline=11,
        lastline=13,
        title={Non-Compliant}
    ]{../code/doxy-rule-001.c}
\end{minipage}
\hfill
\begin{minipage}{0.47\textwidth}
    \lstinputlisting[
        firstline=5,
        lastline=7,
        title={Compliant}
    ]{../code/doxy-rule-001.c}
\end{minipage}

\subsection*{\doxygenRule{}}

A doxygen comment block used for documenting a function shall include an initial section annotated by the \textbackslash brief command. The text for this command shall be a short, preferably one-line, description capturing the core functionality. The \textbackslash brief section shall be followed by an empty line.

\subsubsection*{Examples}

\noindent
\begin{minipage}[t]{\codelstwidth\textwidth}
    \lstinputlisting[
        xleftmargin = 3.4pt,
        firstline = 6,
        lastline = 11,
        title = {Non-Compliant}
    ]{../code/doxy-rule-002-non-compliant.c}
\end{minipage}\hfill
\begin{minipage}[t]{\codelstwidth\textwidth}
    \lstinputlisting[
        xrightmargin = 3.4pt,
        firstline = 6,
        lastline = 12,
        title = {Compliant}
    ]{../code/doxy-rule-002-compliant.c}
\end{minipage}

\subsubsection*{Reasoning}

 Most functions worthy of a Doxygen comment block deserve at least one line of documentation describing functionality and purpose. The \textbackslash brief command is meant to be used for this type of short, concise, documentation. More detailed description can optionally be inserted after the empty line TODO add reference to doxygen rule 003.

\subsection*{\doxygenRule{}}

A doxygen comment block used for documenting a function can in addition to the  \textbackslash brief section have one or more sections with more detailed description of the functionality. Each section of this type be shall annotated with the \textbackslash details command. A \textbackslash details section shall be followed by an empty line.

\subsubsection*{Examples}

\noindent
\begin{minipage}[t]{0.47\textwidth}
    \lstinputlisting[
        xleftmargin = 3.4pt,
        firstline = 11,
        lastline = 13,
        title = {Non-Compliant}
    ]{../code/doxy-rule-001.c}
\end{minipage}\hfill
\begin{minipage}[t]{0.47\textwidth}
    \lstinputlisting[
        xrightmargin=3.4pt,
        firstline=6,
        lastline=20,
        title={Compliant}
    ]{../code/doxy-rules/doxy-rule-003-compliant.c}
\end{minipage}

\noindent
\begin{minipage}[t]{0.47\textwidth}
    \lstinputlisting[
        xleftmargin=3.4pt,
        firstline=15,
        lastline=17,
        title={Non-Compliant}
    ]{../code/doxy-rule-001.c}
\end{minipage}\hfill

\noindent
\begin{minipage}[t]{0.47\textwidth}
    \lstinputlisting[
        xleftmargin=3.4pt,
        firstline=19,
        lastline=21,
        title={Non-Compliant}
    ]{../code/doxy-rule-001.c}
\end{minipage}\hfill

\subsubsection*{Reasoning}

Doxygen supports multiple different formats for the comments block that is the base for the documentation generation. A single format shall be used in a project to improve source code readability. What format to use is somewhat arbitrary.

The format specified by this rule is chosen due to that it closely resembles ordinary C comment blocks. Many editors will understand that this is a Doxygen comment block and highlight the block accordingly, and it is also understood and formatted correctly by many automatic source code formatting tools.



\end{document}
