\documentclass{article}

\usepackage{booktabs}       % For more professional looking tables.
\usepackage{listings}       % For displaying source code.
\usepackage{inconsolata}    % To enable usage of font that mimics Consolas font.
\usepackage{tabularx}  

% Import the
\lstset{
  basicstyle=\ttfamily,     % Change to mono spaced font for source code. 
  tabsize=4,                % Reduce the tab size.
  language=C                % Set C as programming language.
}

\begin{document}

\section{Introduction}

\subsection{Terms and Definitions}

A quick reference with explanations of terms used in this document is provided in Table \ref{tab:terms-and-defs}. Note that the definitions are often C programming language specific and might not be true in other contexts. 

\begin{table}[h]
\begin{tabularx}{\textwidth}{lX}
\toprule
Term & Definition \\
\midrule
allocated storage duration & Object with dynamic storage duration, the storage is created and destroyed upon request. \\
automatic	storage duration &	Object with storage duration inside a block only, from the point of declaration to the end of the block. \\
block & A range of statements enclosed between a pair of braces. \\
object & A location in memory whose content can represent values. \\
static storage duration & Object with storage duration from start to end of program execution. \\ 
storage duration & Determines an objects lifetime. There are three storage durations; allocated, auto, and static. \\
storage class specifier & Specifies storage duration and linkage of objects. There are five specifiers; auto, extern, register, static, and \_Thread\_local. \\
\bottomrule
\end{tabularx}
\caption{Term and definition quick reference}
\label{tab:terms-and-defs}
\end{table}

\section{Rules}

\subsection{Initialize Automatic Variables}

Variables with automatic storage duration shall be initialized before use.

\begin{minipage}{.48\textwidth}
\begin{lstlisting}
int foo(void)
{
	uint32_t i;
	i++;
	return i;
}
\end{lstlisting}
\end{minipage}

\begin{minipage}{.48\textwidth}
\begin{lstlisting}
int foo(void)
{
	uint32_t i;
	i++;
	return i;
}
\end{lstlisting}
\end{minipage}


\begin{minipage}[b]{0.45\linewidth}
\centering
\lstset{language=C,label=SliceExaple}
\begin{lstlisting}[frame=single, numbers=left, mathescape,%
   caption={Example WHILE code snippet with nested scopes.}, label=scopingExample]
x:=1;
if(x>0)
{
x:=5;
while(x>0)
{
   x:=x-1;
}
x:=10;
}
else
{
write x;
}
x:=15;
\end{lstlisting}
\end{minipage}
\hspace{0.5cm}
\begin{minipage}[b]{0.45\linewidth}
\centering
\begin{lstlisting}[frame=single, mathescape,%
   title={Contents of scopesEntered for each block.}, label=scopesEnteredContents]
[0]
[0]
 
[1;0]
[1;0]
 
[2;1;0]
 
[1;0]
 
 
 
[3;0]
 
[0]
\end{lstlisting}
\end{minipage}


\end{document}