\section*{Appendix A Doxygen}

\subsection*{Introduction}

Doxygen is a tool that generates documentation from annotated source files. A software module can be documented directly when developing the code for the module. This workflow makes it more likely that the documentation is kept consistent with the source code.

The documentation can be generated into various formats depending on the needs of the project process. Examples of supported formats are PDF, HTML, RTF, and \LaTeX{}.

Doxygen is commonly used to document modules and functions intended usage in a textual format. But is also possible to generate various visual representations of the elements in the form of graphs and diagrams. These can be used learn the structure of a project or to verify that the implementation is consistent with the design.

The annotations added to the source code are mostly straightforward and human readable. It is possible for developers to digest the documentation directly in the raw source code without need to run the generation process.

\subsection*{Source Code Annotation Rules}

The Doxygen specific annotations are created by incorporating special kind of comment blocks into the source files. These comment blocks can be written in several different ways and still be accepted by Doxygen. This section specifies a set of rules for how the write the comment blocks. The purpose of the rules is; to have consistency, not miss out on required documentation, trouble free coexistence with other tools.

\subsection*{\doxygenRule{}}

Mark Doxygen comment blocks as a C-style comment block, with the difference that there shall be two initial asterisks.

\noindent
\begin{minipage}{0.47\textwidth}
    \lstinputlisting[
        firstline=11,
        lastline=13,
        title={Non-Compliant}
    ]{../code/doxy-rule-001.c}
\end{minipage}
\hfill
\begin{minipage}{0.47\textwidth}
    \lstinputlisting[
        firstline=5,
        lastline=7,
        title={Compliant}
    ]{../code/doxy-rule-001.c}
\end{minipage}

\subsection*{\doxygenRule{}}

Doxygen annotation commands shall start with a backslash.

\subsubsection*{Examples}

\noindent
\begin{minipage}[t]{0.47\textwidth}
    \lstinputlisting[
        xleftmargin = 3.4pt,
        firstline = 3,
        lastline = 5,
        title = {Non-Compliant}
    ]{../code/doxy-rules/doxy-rule-004-non-compliant.c}
\end{minipage}\hfill
\begin{minipage}[t]{0.47\textwidth}
    \lstinputlisting[
        xrightmargin = 3.4pt,
        firstline = 3,
        lastline = 5,
        title = {Compliant}
    ]{../code/doxy-rules/doxy-rule-004-compliant.c}
\end{minipage}


\subsubsection*{Reasoning}

Doxygen accepts two different styles for how to write the annotation commands. A command can start with either a backslash or the at-symbol.
The backslash is chosen to be used because this symbol is part of the basic character set. It is desirable to have the entire source code base written using only characters from the basic character set due that it minimizes the risk of problems with portability and usage of various tools.

\subsection*{\doxygenRule{}}

A source code file with Doxygen annotations shall hold an initial Doxygen comment block where the first command shall be the \textbackslash file command followed by the name of the file.

\subsubsection*{Examples}

The examples assumes that name of the file where the comment block resides is main.c.

\noindent
\begin{minipage}[t]{0.47\textwidth}
    \lstinputlisting[
        xleftmargin = 3.4pt,
        firstline = 1,
        lastline = 3,
        title = {Non-Compliant}
    ]{../code/doxy-rules/doxy-rule-005-non-compliant.c}
\end{minipage}\hfill
\begin{minipage}[t]{0.47\textwidth}
    \lstinputlisting[
        xrightmargin = 3.4pt,
        firstline = 1,
        lastline = 3,
        title = {Compliant}
    ]{../code/doxy-rules/doxy-rule-005-compliant.c}
\end{minipage}

\subsubsection*{Reasoning}

Doxygen will, assuming default settings, only run the documentation process of global objects in a file if the file itself is marked as to be documented. This means that more or less all files in a project should include the \textbackslash file command at the top of the file.

\subsection*{\doxygenRule{}}

A doxygen comment block used for documenting a function shall include an initial section annotated by the \textbackslash brief command. The text for this command shall be a short, preferably one-line, description capturing the core functionality. The \textbackslash brief section shall be followed by an empty line.

\subsubsection*{Examples}

\noindent
\begin{minipage}[t]{\codelstwidth\textwidth}
    \lstinputlisting[
        xleftmargin = 3.4pt,
        firstline = 6,
        lastline = 11,
        title = {Non-Compliant}
    ]{../code/doxy-rule-002-non-compliant.c}
\end{minipage}\hfill
\begin{minipage}[t]{\codelstwidth\textwidth}
    \lstinputlisting[
        xrightmargin = 3.4pt,
        firstline = 6,
        lastline = 12,
        title = {Compliant}
    ]{../code/doxy-rule-002-compliant.c}
\end{minipage}

\subsubsection*{Reasoning}

 Most functions worthy of a Doxygen comment block deserve at least one line of documentation describing functionality and purpose. The \textbackslash brief command is meant to be used for this type of short, concise, documentation. More detailed description can optionally be inserted after the empty line TODO add reference to doxygen rule 003.

\subsection*{\doxygenRule{}}

A doxygen comment block used for documenting a function can in addition to the  \textbackslash brief section have one or more sections with more detailed description of the functionality. Each section of this type be shall annotated with the \textbackslash details command. A \textbackslash details section shall be followed by an empty line.

\subsubsection*{Examples}

\noindent
\begin{minipage}[t]{0.47\textwidth}
    \lstinputlisting[
        xleftmargin = 3.4pt,
        firstline = 11,
        lastline = 13,
        title = {Non-Compliant}
    ]{../code/doxy-rule-001.c}
\end{minipage}\hfill
\begin{minipage}[t]{0.47\textwidth}
    \lstinputlisting[
        xrightmargin=3.4pt,
        firstline=6,
        lastline=20,
        title={Compliant}
    ]{../code/doxy-rules/doxy-rule-003-compliant.c}
\end{minipage}

\noindent
\begin{minipage}[t]{0.47\textwidth}
    \lstinputlisting[
        xleftmargin=3.4pt,
        firstline=15,
        lastline=17,
        title={Non-Compliant}
    ]{../code/doxy-rule-001.c}
\end{minipage}\hfill

\noindent
\begin{minipage}[t]{0.47\textwidth}
    \lstinputlisting[
        xleftmargin=3.4pt,
        firstline=19,
        lastline=21,
        title={Non-Compliant}
    ]{../code/doxy-rule-001.c}
\end{minipage}\hfill

\subsubsection*{Reasoning}

Doxygen supports multiple different formats for the comments block that is the base for the documentation generation. A single format shall be used in a project to improve source code readability. What format to use is somewhat arbitrary.

The format specified by this rule is chosen due to that it closely resembles ordinary C comment blocks. Many editors will understand that this is a Doxygen comment block and highlight the block accordingly, and it is also understood and formatted correctly by many automatic source code formatting tools.

