\subsection*{\codingRule{}}

The intentional null statement i.e.\ a statement that have no effect shall be annotated by an empty macro named DO\_NOTHING. This macro shall be defined in a single place, suitably in a project specific header file.    % ChkTeX 13

\subsubsection*{Examples}

\noindent
\begin{minipage}[t]{\codelstwidth\linewidth}
    \lstinputlisting[
        xleftmargin = 3.4pt,
        firstline = 43,
        lastline = 61,
        title = {Non-Compliant}
        ]{../code/rule-005/rule-005-non-compliant.c}
\end{minipage}
\hfill
\begin{minipage}[t]{\codelstwidth\linewidth}
    \lstinputlisting[
        xrightmargin = 3.4pt,
        linerange = {4-9, 42-61},    % ChkTeX 8
        title = {Compliant}
        ]{../code/rule-005/rule-005-compliant.c}
\end{minipage}

\subsubsection*{Reasoning}

Explicitly stating that there is nothing to do makes it clear that the statement have not been left blank by mistake. Having a standardized way to indicate this with a macro improves readability, since every occurrence of this type will look the same throughout all project source files.